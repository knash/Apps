
\documentclass[12pt]{article}
%\usepackage[
%backend=bibtex,
%citestyle=ieee
%]{biblatex}
%\addbibresource{RI2020.bib}
%\renewbibmacro{in:}{}
%% Define some new colors
\usepackage{xcolor}

\usepackage{scrextend}


%% Page and text formatting
\usepackage[left=1.0in, right=1.0in, top=1.0in, bottom=1.0in]{geometry} % margins
\usepackage{setspace}
\singlespacing % No more than 6 lines of text per inch
\usepackage{amsmath, amsfonts}
\usepackage[T1]{fontenc}
\usepackage{times}
\usepackage{graphics}
\usepackage{makecell}
\usepackage{tabularx}
\usepackage{indentfirst}
\usepackage[normalem]{ulem}
%% Set up hyperlinks
\usepackage[linktoc=page]{hyperref}
\hypersetup{
  colorlinks = true,
  urlcolor = blue,
  citecolor = blue,
  linkcolor = blue,
}
\usepackage{doi}

%% Reduce whitespace around lists (globally)
\usepackage{enumitem}
\setlist[itemize]{nosep,left=0pt}
\setlist[enumerate]{nosep}
\setlist[description]{nosep}

%% Formatting for sections

\usepackage{sectsty}
\sectionfont{\fontsize{22}{25}\selectfont}
\subsectionfont{\fontsize{15}{25}\selectfont}




\begin{document}

\fontsize{10}{12}\selectfont

\linespread{1.3}
\fontfamily{ptm}\selectfont

\vspace{2mm}

\vspace{1mm}

\noindent{\fontsize{15}{17}\selectfont Kevin Nash \\ }

\noindent\begin{tabular*}{\textwidth}{@{\extracolsep{\fill}}l l}
Research Associate & Email: \href{mailto:knash@physics.rutgers.edu}{knash@physics.rutgers.edu} \\
Rutgers University & Phone: (434) 760-1424\\
Department of Physics \& Astronomy \\
136 Frelinghuysen Rd\\
Piscataway, NJ 08854\\
\hline
\\
\\
\end{tabular*}




\section*{Research Statement}


The standard model of particle physics has been an extremely successful tool,
but we know that it remains incomplete.  If new physics is discoverable at the LHC,
then we will need to use the most sensitive tools and combine measurements from
as many signatures as possible to find it.  Possibly the most room for improvement
can be gained from searches in the
all hadronic final state due to the difficulty in reducing the QCD multijet
background.  These searches can greatly benefit from cutting-edge
selection techniques, and of these, perhaps the most exciting is the tagging
of merged heavy resonances.  Merged heavy resonances arise in BSM physics
searches for massive gauge bosons, heavy quark partners, SUSY partners,
Kaluza-Klein excitations, and others.  Classically, analyses that identify
hadronic signatures (for example from a top decay) would not be as sensitive
as the leptonic decay mode, due to the extremely difficult problem of reducing
the hadronic backgrounds.

A boosted heavy resonance decay can be reconstructed as a single wide jet with
a characteristic substructure, and heavy resonance tagging techniques use this
substructure to reduce the hadronic background.  Generally, tagging techniques
discriminate boosted resonances using analytical variables such as the ``groomed''
mass of the jet which selects a jet mass consistent with the progenitor particle,
a shape criterion that selects an energy deposition pattern consistent with the
number or expected cores, and a selection of the heavy flavor content within the
jet.  These methods have had much success, but recent developments in machine-learned pattern recognition improve the sensitivity far beyond what is achieved
by analytical variable selections.  First, I will discuss my historical contributions to
merged jet tagging, then my work in porting convolutional neural network
(CNN) image recognition to jet tagging and the rich
physics analysis program that these methods make possible, and finally my role in
the outer tracker hardware program in preparation of the HL-LHC upgrade.

My introduction to LHC physics research was the application of the first tagging techniques
for merged top jets.  I applied the methods that were pioneered in the $\mathrm{Z' \to t\overline{t}}$
all hadronic
search \cite{Chatrchyan:2013lca}, to the $\mathrm{W' \to t\overline{b}}$ search
\cite{Khachatryan:2015edz} (see also the run 2 reboot in Ref.~\cite{Sirunyan:2017ukk}).
I was the first to use the N-subjettiness and subjet b-tagging algoritms for top-jet identification, which
proved to be much more effective in reducing the
QCD-multijet background than the methods used previously.
I extracted the scale factor for this new top tagger by studying a highly
pure sample of semileptonic $\mathrm{t\overline{t}}$ (leading to the JME-13-007 \cite{CMS:2014fya} analysis).
 I then applied this new top tagging method to the hadronic $\mathrm{b^* \to tW}$
 search \cite{Khachatryan:2015mta}, and then additionally to
the $\mathrm{W' \to qQ \to tHb}$ search \cite{Sirunyan:2018fki}.  Motivated by extensive
studies in the theoretical community \cite{Kasieczka:2017nvn,Macaluso:2018tck}, I
employed the latest advances in machine learning to produce a
CNN tagger for CMS, ``$\mathrm{imageTop}$''.  The improvements that I made on top of the previous
CNN studies include an adaptive zoom, particle flow identification ``colors'', flavor
discrimination, and mass decorrelation.  This tagger debuted in Ref.~\cite{CMS:2019gpd},
and shows an improvement of nearly a {\bf factor of ten} in background rejection compared to the standard
CMS top-tagger, and is the most sensitive mass decorrelated tagger in CMS.

The $\mathrm{imageTop_{MD}}$ network offers a substantial improvement over other tagging
methods, so I then expanded to include additional physics objects ($\mathrm{imageX_{MD}}$) and
moved to the physics analysis implementation.  I designed a data-driven background estimate
that is capable of predicting a generic final state to create an ``analysis factory'',
which takes advantage of the kinematic decorrelation of the $\mathrm{imageX_{MD}}$ network.  In order
to commission the $\mathrm{imageTop_{MD}}$ tagger and analysis factory, I rebooted
the $\mathrm{W' \to qQ \to tHb}$ search using now the full CMS run 2 data. This analysis
is now post-approval, and the next steps will involve multiple extensions of the tagger and
analysis methods.  We are investigating the sensitivity of the ImageMD network when applied to several
exotic jet signatures.  In the case of a boosted resonance decaying to two bosons (For example a WED-motivated light radion),
we expect a merged diboson jet (for example $\mathrm{WW \to qqqq}$ or $\mathrm{Z\gamma \to qq\gamma}$ within a single jet).
In the WW case, we expect a dedicated neural network training to give a substantial improvement in
such a feature-dense jet, and in the $\mathrm{Z\gamma}$ case, we expect the photon and neutral pf
candidate colors to improve signal efficiency in the case of a non-isolated
photon.  Additionally, there is room to expand the aforementioned
$\mathrm{W' \to qQ \to tHb}$ search in interesting ways.  The same final state here is
reproduced by the $\mathrm{W' \to HH^{\pm} \to tHb}$ signature \cite{Dobrescu:2015yba}.
However, here there is the possibility of a boosted $\mathrm{H^{\pm}}$ which would produce a
merged top jet with an additional hard b quark.
Also, the same theoretical framework allows for a robust set of analyses the
analysis with the inclusion of the additional $\mathrm{W'}$ decays
$\mathrm{W' \to H^{0}H^{\pm} \to tttb}$ or $\mathrm{tbbb}$, which would similarly cover an element of phase space
that necessitates a dedicated boosted bb or tt jet tagger.
These additional merged jets are prime candidates for the $\mathrm{imageX_{MD}}$
network.  In preliminary studies, the exotic taggers show a large improvement over the
conventional methods, and the analysis factory produces an impressive background
closure with only minor tweaking.

The main challenge in producing BSM taggers is the data validation, given that there
is usually no acceptable SM analogue, and a machine-learned tagger necessitates a
high level of confidence, given the lack of physical intuition of the learned features.
Therefore, I am investigating methods of validation through event superposition
which can be used to create BSM jet analogues in CMS data.  Although the network is
very sensitive in the current form, we are additionally looking toward future improvements.
Potentially $\mathrm{image_{MD}}$ network can be improved by expanding the pf candidate inputs to
instead include lower-level detector subsystems, and by converting the pixelized inputs
to particle-based inputs (a GCNN for example).  With my technical experience from implementing the
$\mathrm{image_{MD}}$ network, and leadership experience from my
B2G:RES and JMAR subgroup convenerships, position to play a leading role in the
field of hadronic searches, which will carry into the HL-LHC era.

Although these dedicated methods are
extremely powerful, the network can also be used in a generic way.
We are currently investigating the application of the network as an unsupervised
anomaly detector through the porting of the architecture to an autoencoder.
With this network, an anomaly would show up as a poorly reconstructed jet image.
Additionally, the anomaly detection performance is bolstered by a careful determination of the latent space pdf by using autoregressive flows.
The anomaly is then detected as an unlikely latent space configuration.

Looking towards the future, the HL-LHC offers benefits to analysis sensitivity
but also unique challenges that need to be addressed with new analysis methods
and detector hardware improvements.  On the analysis side, the dense environment
will certainly only make machine-learned methods like $\mathrm{image_{MD}}$ more
relevant.  From the hardware side, I am helping to prepare for the HL-LHC
upgrade through the outer tracker silicon module prototyping.
Specifically, characterizing the PS-module, which correlates hits from two closely
spaced silicon layers such that a fast $p_\mathrm{T}$ measurement can be made.  I contributed to
this effort by characterizing the prototype assembly for the silicon pixel portion
of the PS-module.

My work with the PS-module started from studying the first bare readout chip, the MPA-light.
Here, I was in charge of wirebonding and software development.
The task involved the initial testing with prototype PCBs and firmware, which offered a unique challenge.
The bare readout chip allowed me to
write the first functional DAQ system for data taking.
The next step was to test the MaPSA-light, which consisted of eight MPA chips
that have been bump bonded to a pixelized silicon sensor.
I was in charge of the MaPSA-light testbeam campaign as the spokesperson
for experiment T-1209, which was performed using the Fermilab Main Injector.
I also added to
the analysis effort through the unfolded time efficiency measurement using a convolved
fit to the timing response spectrum.
After the MaPSA-light, we created a ``micro module'' from two closely spaced MaPSA-light
assemblies.  This allows the correlation of hits from the two devices in an analogous
way to the full PS-module.  This lead to additional testbeam campaigns where
the correlated hit pairs could be studied in real data.
The data collected during the multiple testbeams were
used in various analyses contributing to the Phase-II outer tracker TDR.

My experience contributing to the upgrade
project has given me a better understanding of the breadth of a full physics analysis,
from achieving the first readout on an experimental detector to a publishable result.
My upgrade duties continue, leading toward the final MPA (and SSA) chip software integration
with the full Phase-II outer tracker software framework.
Currently, I am finalizing the software for the half PS-module at Fermilab,
the latest evolution of the OT hardware.
My work characterizing
the very first prototype MPA modules has been an exciting introduction to HL-LHC
upgrade hardware and the experience that I have gained will prove valuable in
establishing further hardware contributions.
\normalem

\begin{flushleft}
\begin{thebibliography}{1}


 \bibitem{Chatrchyan:2013lca} \textit{Search for New Physics using the $\mathrm{t\bar{t}}$ Invariant Mass Distribution in pp Collisions at $\sqrt{s}$=8  TeV},\\
 The CMS Collaboration, \textbf{Phys. Rev. Lett.},{\doi{10.1103/PhysRevLett.111.211804}}, 2013

 \bibitem{Khachatryan:2015edz} \textit{Search for $\mathrm{W' \to tb}$ in proton-proton collisions at $\sqrt{s} = $ 8 TeV},\\
 The CMS Collaboration, \textbf{JHEP}, \doi{10.1007/JHEP02(2016)122}, 2016

 \bibitem{Sirunyan:2017ukk} \textit{Searches for $\mathrm{W'}$ bosons decaying to a top quark and a bottom quark in proton-proton collisions at 13 TeV}, \\
  The CMS Collaboration, \textbf{JHEP}, \doi{10.1007/JHEP08(2017)029}, 2017



 \bibitem{CMS:2014fya}
 \textit{Boosted Top Jet Tagging at CMS},\\
 The CMS Collaboration, \textbf{CMS Physics Analysis Summry}, \href{http://cms-physics.web.cern.ch/cms-physics/public/JME-13-007-pas.pdf}{{\color{blue}\underline{CMS PAS JME-13-007}}}, 2014


 \bibitem{Khachatryan:2015mta} \textit{Search for the production of an excited bottom quark decaying to $\mathrm{tW}$ in proton-proton collisions at $ \sqrt{s}=8 $ TeV},\\
The CMS Collaboration, \textbf{JHEP}, \doi{10.1007/JHEP01(2016)166}, 2016



 \bibitem{Sirunyan:2018fki} \textit{Search for a $\mathrm{W'}$ boson decaying to a vector-like quark and a top or bottom quark in the all-jets final state},\\
The CMS Collaboration, \textbf{JHEP}, \doi{10.1007/JHEP03(2019)127}, 2019

 \bibitem{Kasieczka:2017nvn} \textit{Deep-learning Top Taggers or The End of QCD?},\\
 Kasieczka, Plehn et al, \textbf{JHEP}, \doi{10.1007/JHEP05(2017)006}, 2017

 \bibitem{Macaluso:2018tck} \textit{Pulling Out All the Tops with Computer Vision and Deep Learning},\\
 Macaluso, Sebastian and Shih, David, \textbf{JHEP}, \doi{10.1007/JHEP10(2018)121}, 2018


 \bibitem{CMS:2019gpd} \textit{Identification of heavy, energetic, hadronically decaying particles using machine-learning techniques},\\
 The CMS Collaboration, \textbf{JINST}, \doi{10.1088/1748-0221/15/06/P06005}, 2020


  \bibitem{Dobrescu:2015yba} \textit{Heavy Higgs bosons and the 2 TeV $\mathrm{W'}$ boson},\\
  Dobrescu, Bogdan A. and Liu, Zhen, \textbf{JHEP}, \doi{10.1007/JHEP10(2015)118}, 2015

\end{thebibliography}
\end{flushleft}


%\bibliography{RI2020}
%\bibliographystyle{IEEEtran}
\end{document}
